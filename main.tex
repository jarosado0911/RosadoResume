\documentclass[10pt, letterpaper]{article}

% Packages:
\usepackage[
    ignoreheadfoot, % set margins without considering header and footer
    top=2 cm, % seperation between body and page edge from the top
    bottom=2 cm, % seperation between body and page edge from the bottom
    left=2 cm, % seperation between body and page edge from the left
    right=2 cm, % seperation between body and page edge from the right
    footskip=1.0 cm, % seperation between body and footer
    % showframe % for debugging 
]{geometry} % for adjusting page geometry
\usepackage{titlesec} % for customizing section titles
\usepackage{tabularx} % for making tables with fixed width columns
\usepackage{array} % tabularx requires this
\usepackage[dvipsnames]{xcolor} % for coloring text
\definecolor{primaryColor}{RGB}{0, 0, 0} % define primary color
\usepackage{enumitem} % for customizing lists
\usepackage{fontawesome5} % for using icons
\usepackage{amsmath} % for math
\usepackage[
    pdftitle={James Rosado's CV},
    pdfauthor={James Rosado},
    pdfcreator={LaTeX with RenderCV},
    colorlinks=true,
    urlcolor=primaryColor
]{hyperref} % for links, metadata and bookmarks
\usepackage[pscoord]{eso-pic} % for floating text on the page
\usepackage{calc} % for calculating lengths
\usepackage{bookmark} % for bookmarks
\usepackage{lastpage} % for getting the total number of pages
\usepackage{changepage} % for one column entries (adjustwidth environment)
\usepackage{paracol} % for two and three column entries
\usepackage{ifthen} % for conditional statements
\usepackage{needspace} % for avoiding page brake right after the section title
\usepackage{iftex} % check if engine is pdflatex, xetex or luatex

% Ensure that generate pdf is machine readable/ATS parsable:
\ifPDFTeX
    \input{glyphtounicode}
    \pdfgentounicode=1
    \usepackage[T1]{fontenc}
    \usepackage[utf8]{inputenc}
    \usepackage{lmodern}
\fi

\usepackage{charter}

% Some settings:
\raggedright
\AtBeginEnvironment{adjustwidth}{\partopsep0pt} % remove space before adjustwidth environment
\pagestyle{empty} % no header or footer
\setcounter{secnumdepth}{0} % no section numbering
\setlength{\parindent}{0pt} % no indentation
\setlength{\topskip}{0pt} % no top skip
\setlength{\columnsep}{0.15cm} % set column seperation
\pagenumbering{gobble} % no page numbering

\titleformat{\section}{\needspace{4\baselineskip}\bfseries\large}{}{0pt}{}[\vspace{1pt}\titlerule]

\titlespacing{\section}{
    % left space:
    -1pt
}{
    % top space:
    0.3 cm
}{
    % bottom space:
    0.2 cm
} % section title spacing

\renewcommand\labelitemi{$\vcenter{\hbox{\small$\bullet$}}$} % custom bullet points
\newenvironment{highlights}{
    \begin{itemize}[
        topsep=0.10 cm,
        parsep=0.10 cm,
        partopsep=0pt,
        itemsep=0pt,
        leftmargin=0 cm + 10pt
    ]
}{
    \end{itemize}
} % new environment for highlights


\newenvironment{highlightsforbulletentries}{
    \begin{itemize}[
        topsep=0.10 cm,
        parsep=0.10 cm,
        partopsep=0pt,
        itemsep=0pt,
        leftmargin=10pt
    ]
}{
    \end{itemize}
} % new environment for highlights for bullet entries

\newenvironment{onecolentry}{
    \begin{adjustwidth}{
        0 cm + 0.00001 cm
    }{
        0 cm + 0.00001 cm
    }
}{
    \end{adjustwidth}
} % new environment for one column entries

\newenvironment{twocolentry}[2][]{
    \onecolentry
    \def\secondColumn{#2}
    \setcolumnwidth{\fill, 4.5 cm}
    \begin{paracol}{2}
}{
    \switchcolumn \raggedleft \secondColumn
    \end{paracol}
    \endonecolentry
} % new environment for two column entries

\newenvironment{threecolentry}[3][]{
    \onecolentry
    \def\thirdColumn{#3}
    \setcolumnwidth{, \fill, 4.5 cm}
    \begin{paracol}{3}
    {\raggedright #2} \switchcolumn
}{
    \switchcolumn \raggedleft \thirdColumn
    \end{paracol}
    \endonecolentry
} % new environment for three column entries

\newenvironment{header}{
    \setlength{\topsep}{0pt}\par\kern\topsep\centering\linespread{1.5}
}{
    \par\kern\topsep
} % new environment for the header

\newcommand{\placelastupdatedtext}{% \placetextbox{<horizontal pos>}{<vertical pos>}{<stuff>}
  \AddToShipoutPictureFG*{% Add <stuff> to current page foreground
    \put(
        \LenToUnit{\paperwidth-2 cm-0 cm+0.05cm},
        \LenToUnit{\paperheight-1.0 cm}
    ){\vtop{{\null}\makebox[0pt][c]{
        \small\color{gray}\textit{Last updated in September 2024}\hspace{\widthof{Last updated in September 2024}}
    }}}%
  }%
}%

% save the original href command in a new command:
\let\hrefWithoutArrow\href

% new command for external links:


\begin{document}
    \newcommand{\AND}{\unskip
        \cleaders\copy\ANDbox\hskip\wd\ANDbox
        \ignorespaces
    }
    \newsavebox\ANDbox
    \sbox\ANDbox{$|$}

    \begin{header}
        \fontsize{25 pt}{25 pt}\selectfont James M. Rosado

        \vspace{5 pt}

        \normalsize
        \mbox{Washington, D.C. 20009}%
        \kern 5.0 pt%
        \AND%
        \kern 5.0 pt%
        \mbox{\hrefWithoutArrow{mailto:jarosado0911@gmail.com}{jarosado0911@gmail.com}}%
        \kern 5.0 pt%
        \AND%
        \kern 5.0 pt%
        %\mbox{\hrefWithoutArrow{tel:856-419-2657}{856-419-2657}}%
        %\kern 5.0 pt%
        %\AND%
        \kern 5.0 pt%
        \mbox{\hrefWithoutArrow{https://github.com/jarosado0911}{GitHub}}%
        \kern 5.0 pt%
        \AND%
        \kern 5.0 pt%
        \mbox{\hrefWithoutArrow{www.linkedin.com/in/rosado-james-3239b2119}{linkedin}}%
        \kern 5.0 pt%
        \AND%
        \kern 5.0 pt%
        \mbox{\hrefWithoutArrow{https://sites.google.com/view/james-rosado-site2}{Google}}%
    \end{header}

    \vspace{5 pt - 0.3 cm}
    \section{Summary}        
        \begin{onecolentry}
            Applied Research Mathematician and Software Developer with over a decade of experience spanning academic research, national security, and computational neuroscience. Holds active Top Secret/SCI clearance and a Ph.D. in Mathematics from Temple University. Proven expertise in mathematical modeling, algorithm design, high-performance computing, and AI/ML systems integration using Python, C/C++, MPI, and OpenMP. Demonstrated success delivering impactful software across government and academic sectors, including VR-based neuron simulations, graph-based query systems, and machine learning recommendation engines. Recognized with multiple awards for software engineering excellence at the NSA. Published author in leading journals and contributor to open-source neuroscience tools. Experienced educator and mentor with a commitment to innovation, cross-disciplinary collaboration, and scientific communication.
        \end{onecolentry}
        \vspace{0.2 cm}    
\section{Education}
    \begin{twocolentry}{Sep. 2017 – June 2022}
        \textbf{Temple University}, Ph.D. in Mathematics
    \end{twocolentry}
    \vspace{0.1 cm}
    \begin{twocolentry}{Sep. 2012 – June 2016}
        \textbf{Rowan University}, M.A. in Mathematics
    \end{twocolentry}
    \vspace{0.1 cm}
    \begin{twocolentry}{Sep. 2008 – Sep. 2009}
        \textbf{NJ Department of Education}, Teacher of Mathematics Certificate
    \end{twocolentry}
    \vspace{0.1 cm}
    \begin{twocolentry}{Sep. 2003 – June 2007}
        \textbf{Rutgers University}, B.S. in Electrical and Computer Engineering
    \end{twocolentry}
    \vspace{0.1 cm}
    \begin{twocolentry}{Sep. 1999 – June 2003}
        \textbf{Pitman High School}, High School Diploma
    \end{twocolentry}

   \section{Experience}

\begin{twocolentry}{
    Sep. 2022 – Present
}
    \textbf{Applied Mathematician}, U.S. Government / Department of Defense / NSA -- Maryland, USA
\end{twocolentry}

\vspace{0.10 cm}
\begin{onecolentry}
    \begin{highlights}
        \item Conducts applied research and software development across various NSA branches with active TS/SCI clearance
        \item Developed graph-based query algorithms; collaborated with LLNL on integration with in-house software
        \item Built ML-based article recommendation systems using Python, Scikit-learn, TensorFlow, and PyTorch
        \item Received multiple agency awards for software engineering and machine learning work
    \end{highlights}
\end{onecolentry}

\vspace{0.2 cm}

\begin{twocolentry}{
    Summer 2021
}
    \textbf{Graduate Mathematics Intern}, NSA Graduate Mathematics Program -- Maryland, USA
\end{twocolentry}

\vspace{0.10 cm}
\begin{onecolentry}
    \begin{highlights}
        \item Researched detection of synthesized speech using bispectrum analysis and topological data analysis
        \item Implemented and evaluated binary classifiers and neural networks for speech classification
        \item Co-authored internal paper and briefed findings to NSA leadership including the Director
    \end{highlights}
\end{onecolentry}

\vspace{0.2 cm}

\begin{twocolentry}{
    Sep. 2019 – Sep. 2022
}
    \textbf{Researcher}, Center for Computational Mathematics \& Modeling (C2M2), Temple University -- Philadelphia, PA
\end{twocolentry}

\vspace{0.10 cm}
\begin{onecolentry}
    \begin{highlights}
        \item Developed Neuro-VISOR, a real-time VR simulation platform for neuronal networks using C\#, Unity, and PDE solvers
        \item Created Python mesh generation software using parallel transport and 1D neuronal geometries
        \item Enabled real-time interaction of synaptic and electrical behavior in virtual environments used in classrooms
    \end{highlights}
\end{onecolentry}

\vspace{0.2 cm}

\begin{twocolentry}{
    Sep. 2018 – Sep. 2022
}
    \textbf{Researcher}, Queisser Research Group, Temple University -- Philadelphia, PA
\end{twocolentry}

\vspace{0.10 cm}
\begin{onecolentry}
    \begin{highlights}
        \item Developed HPC pipelines using MPI, FEM/FVM methods for calcium simulations under TMS
        \item Ran large-scale simulations on SDSC EXPANSE/COMET through NSF XSEDE awards (over 1.8M core-hours)
        \item Mentored students and collaborated across Temple, University of Minnesota, and University of Freiburg
    \end{highlights}
\end{onecolentry}

\vspace{0.2 cm}

\begin{twocolentry}{
    Sep. 2017 – June 2022
}
    \textbf{Graduate Teaching Assistant}, Temple University -- Philadelphia, PA
\end{twocolentry}

\vspace{0.10 cm}
\begin{onecolentry}
    \begin{highlights}
        \item Instructed undergraduate mathematics courses, designed exams, and mentored students in research
        \item Volunteered for MCC tutoring and led SIAM workshops and student research talks
    \end{highlights}
\end{onecolentry}

\vspace{0.2 cm}

\begin{twocolentry}{
    Sep. 2016 – June 2017
}
    \textbf{Adjunct Professor}, Rowan University -- Glassboro, NJ
\end{twocolentry}

\vspace{0.10 cm}
\begin{onecolentry}
    \begin{highlights}
        \item Taught evening Calculus I courses and maintained active student engagement through Canvas
        \item Designed and graded assessments; provided regular student support and feedback
    \end{highlights}
\end{onecolentry}

\vspace{0.2 cm}

\begin{twocolentry}{
    Sep. 2024 – Dec. 2024
}
    \textbf{Adjunct Professor (Online)}, Rowan University -- Glassboro, NJ
\end{twocolentry}

\vspace{0.10 cm}
\begin{onecolentry}
    \begin{highlights}
        \item Administered and online mathematics course
        \item Designed and graded assessments; provided regular student support and feedback via Canvas
    \end{highlights}
\end{onecolentry}

\vspace{0.2 cm}

\begin{twocolentry}{
    Jan. 2025 – May 2025
}
    \textbf{Adjunct Professor}, Towson University -- Towson, MD
\end{twocolentry}

\vspace{0.10 cm}
\begin{onecolentry}
    \begin{highlights}
        \item Taught an evening course in elementary mathematics
        \item Designed and graded assessments; provided regular student support and feedback via Blackboard
    \end{highlights}
\end{onecolentry}

\vspace{0.2 cm}

\begin{twocolentry}{
    Sep. 2009 – June 2017
}
    \textbf{High School Mathematics Teacher}, Clearview Regional School District -- Mullica Hill, NJ
\end{twocolentry}

\vspace{0.10 cm}
\begin{onecolentry}
    \begin{highlights}
        \item Taught 9th–12th grade math, co-taught special education classes, and led night/homebound instruction
        \item Developed interactive lessons aligned with NJ Core Content Standards and IEP accommodations
    \end{highlights}
\end{onecolentry}

\vspace{0.2 cm}

\begin{twocolentry}{
    Sep. 2012 – June 2016
}
    \textbf{Mathematics Researcher}, Rowan University -- Glassboro, NJ
\end{twocolentry}

\vspace{0.10 cm}
\begin{onecolentry}
    \begin{highlights}
        \item Developed and published finite frame partitioning algorithms using Mathematica and complex analysis
        \item Presented work at the Joint Mathematics Meetings (JMM) and STEM symposia
    \end{highlights}
\end{onecolentry}

\vspace{0.2 cm}

\begin{twocolentry}{
    Sep. 2011 – June 2016
}
    \textbf{Private Tutor}, Gloucester County, NJ
\end{twocolentry}

\vspace{0.10 cm}
\begin{onecolentry}
    \begin{highlights}
        \item Delivered one-on-one instruction from grade school through high school mathematics
        \item Coordinated with parents and used manipulatives to reinforce student understanding
    \end{highlights}
\end{onecolentry}

\section{Publications}

\begin{samepage}
    \begin{twocolentry}{
        Dec 2023
    }
        \textbf{Neuronal Resilience and Calcium Signaling Pathways in the Context of Synapse Loss and Calcium Leaks}
    \end{twocolentry}

    \vspace{0.10 cm}

    \begin{onecolentry}
        \mbox{Rosado, J. M.}, \mbox{Borole, P. R.}, \mbox{Neal, M.}, \mbox{Queisser, G.}

        \vspace{0.10 cm}

        \href{https://doi.org/10.1137/22M1512131}{SIAM Journal on Applied Mathematics, Vol. 83, Issue 6, pp. 2418–2442}
    \end{onecolentry}
\end{samepage}

\vspace{0.3 cm}

\begin{samepage}
    \begin{twocolentry}{
        Nov 2021
    }
        \textbf{Multi-scale Modeling Toolbox for Single Neuron and Subcellular Activity Under TMS}
    \end{twocolentry}

    \vspace{0.10 cm}

    \begin{onecolentry}
        \mbox{Rosado, J. M.}, \mbox{Shirinpour, S.}, \mbox{Hananeia, N.}, \mbox{Tran, H.}, \mbox{Galanis, C.}, \mbox{Vlachos, A.}, \mbox{Jedlicka, P.}, \mbox{Queisser, G.}, \mbox{Opitz, A.}

        \vspace{0.10 cm}

        \href{https://doi.org/10.1016/j.brs.2021.09.024}{Brain Stimulation, Vol. 14, Issue 6, pp. 1470–1482}
    \end{onecolentry}
\end{samepage}

\vspace{0.3 cm}

\begin{samepage}
    \begin{twocolentry}{
        Apr 2022
    }
        \textbf{Calcium Modeling of Spine Apparatus-containing Human Dendritic Spines}
    \end{twocolentry}

    \vspace{0.10 cm}

    \begin{onecolentry}
        \mbox{Rosado, J. M.}, \mbox{Bui, V. D.}, \mbox{Haas, C. A.}, \mbox{Beck, J.}, \mbox{Queisser, G.}, \mbox{Vlachos, A.}

        \vspace{0.10 cm}

        \href{https://doi.org/10.1371/journal.pcbi.1010014}{PLOS Computational Biology, April 2022}
    \end{onecolentry}
\end{samepage}

\vspace{0.3 cm}

\begin{samepage}
    \begin{twocolentry}{
        Mar 2017
    }
        \textbf{Partitions of Equiangular Tight Frames}
    \end{twocolentry}

    \vspace{0.10 cm}

    \begin{onecolentry}
        \mbox{Rosado, J. M.}, \mbox{Nguyen, H. D.}, \mbox{Cao, L.}

        \vspace{0.10 cm}

        \href{https://doi.org/10.1016/j.laa.2016.12.016}{Linear Algebra and its Applications, Vol. 526, pp. 95–120}
    \end{onecolentry}
\end{samepage}

\vspace{0.3 cm}

\begin{samepage}
    \begin{twocolentry}{
        Aug 2015
    }
        \textbf{A Table of Definite Integrals from the Marriage of Power and Fourier Series}
    \end{twocolentry}

    \vspace{0.10 cm}

    \begin{onecolentry}
        \mbox{Rosado, J. M.}, \mbox{Osler, T.}

        \vspace{0.10 cm}

        \href{https://scientia.wpunj.edu/issue/view/24}{Scientia, Vol. 26, pp. 77–82}
    \end{onecolentry}
\end{samepage}

\vspace{0.3 cm}

\begin{samepage}
    \begin{twocolentry}{
        May 2022
    }
        \textbf{Ultrastructural Neuronal Modeling of Calcium Dynamics Under TMS} (Doctoral Dissertation)
    \end{twocolentry}

    \vspace{0.10 cm}

    \begin{onecolentry}
        \mbox{Rosado, James Michael}

        \vspace{0.10 cm}

        \href{https://digital.library.temple.edu/digital/collection/p245801coll10/id/620358}{Temple University Libraries}
    \end{onecolentry}
\end{samepage}

\vspace{0.3 cm}

\begin{samepage}
    \begin{twocolentry}{
        June 2016
    }
        \textbf{Partitions of Finite Frames} (Master's Thesis)
    \end{twocolentry}

    \vspace{0.10 cm}

    \begin{onecolentry}
        \mbox{Rosado, James Michael}

        \vspace{0.10 cm}

        \href{https://www.proquest.com/openview/845a0ac5c3c9c4b7ebd6d1669d1c83ef}{ProQuest Theses and Dissertations, Rowan University}
    \end{onecolentry}
\end{samepage}



    
   \section{Projects}

\begin{samepage}
    \begin{twocolentry}{
        2018 – 2022
    }
        \textbf{Neuro-VISOR: Virtual Interactive Simulation of Reality}
    \end{twocolentry}

    \vspace{0.10 cm}
    
    \begin{onecolentry}
        Developed at Temple University's C2M2 Lab for real-time VR simulation of neuronal dynamics.
        \vspace{0.10 cm}

        \href{https://github.com/c2m2/Neuro-VISOR}{github.com/c2m2/Neuro-VISOR}
    \end{onecolentry}
\end{samepage}

\vspace{0.3 cm}

\begin{samepage}
    \begin{twocolentry}{
        2018 – 2019
    }
        \textbf{NeMo-TMS: Neuron Modeling for TMS}
    \end{twocolentry}

    \vspace{0.10 cm}
    
    \begin{onecolentry}
        Multi-scale modeling toolbox to simulate transcranial magnetic stimulation effects on neurons.

        \vspace{0.10 cm}
        
        \href{https://github.com/OpitzLab/NeMo-TMS}{github.com/OpitzLab/NeMo-TMS}
    \end{onecolentry}
\end{samepage}

\vspace{0.3 cm}

\begin{samepage}
    \begin{twocolentry}{
        2021 – 2023
    }
        \textbf{CalcSim: Calcium Dynamics Simulator}
    \end{twocolentry}

    \vspace{0.10 cm}
    
    \begin{onecolentry}
        MATLAB-based simulator for modeling intracellular calcium dynamics in single neurons.

        \vspace{0.10 cm}
        
        \href{https://github.com/NeuroBox3D/CalcSim}{github.com/NeuroBox3D/CalcSim}
    \end{onecolentry}
\end{samepage}

\vspace{0.3 cm}

\begin{samepage}
    \begin{twocolentry}{
        2022 – Present
    }
        \textbf{PythonNeuronMeshes}
    \end{twocolentry}

    \vspace{0.10 cm}
    
    \begin{onecolentry}
        Python software for generating surface meshes from 1D neuron morphology using parallel transport methods.

        \vspace{0.10 cm}
        
        \href{https://github.com/jarosado0911/PythonNeuronMeshes}{github.com/jarosado0911/PythonNeuronMeshes}
    \end{onecolentry}
\end{samepage}


    
\section{Technologies}

\begin{onecolentry}
    \textbf{Languages:} C++, C, Python, Java, C\#, MATLAB, LUA, Shell scripting, LaTeX
\end{onecolentry}

\vspace{0.2 cm}

\begin{onecolentry}
    \textbf{Technologies:} Unity3D, Git, GitHub, GitLab, Atlassian Bitbucket, MPI, OpenMP, Blackboard, Canvas, PowerSchool, PowerTeacher
\end{onecolentry}

\vspace{0.2 cm}

\begin{onecolentry}
    \textbf{Libraries \& Frameworks:} TensorFlow, PyTorch, Scikit-learn, word2vec, uG4 (unstructured grid framework)
\end{onecolentry}

\vspace{0.2 cm}

\begin{onecolentry}
    \textbf{Tools:} Zoom, Skype, Geometer's Sketchpad, TI-84 Graphing Calculator
\end{onecolentry}

\vspace{0.2 cm}

\begin{onecolentry}
    \textbf{Operating Systems:} Linux (RedHat, CentOS), macOS, Windows
\end{onecolentry}

\section{Awards}

\begin{onecolentry}
    \textbf{DoD/NSA Awards:}
\end{onecolentry}

\vspace{0.1 cm}

\begin{onecolentry}
    2025 Monetary Award – For contributing to improved software performance - \$2,000
\end{onecolentry}

\begin{onecolentry}
    2024 Monetary Award – For developing and implementing article recommendation systems - \$3,000
\end{onecolentry}

\begin{onecolentry}
    2024 Performance Coin Award – For work completed on ML-based recommendation systems
\end{onecolentry}

\begin{onecolentry}
    2023 Time Off Award – For advanced coding practices in software engineering - 16 hrs.
\end{onecolentry}

\begin{onecolentry}
    2023 Monetary Award – For performance involving advanced query-based algorithms - \$2,000
\end{onecolentry}

\vspace{0.3 cm}

\begin{onecolentry}
    \textbf{National Science Foundation HPC Awards:}
\end{onecolentry}

\vspace{0.1 cm}

\begin{onecolentry}
    2022 XSEDE Compute Research Award – \$2,440
\end{onecolentry}

\begin{onecolentry}
    2021 XSEDE Compute Research Award – \$2,440
\end{onecolentry}

\begin{onecolentry}
    2020 XSEDE Compute Research Award – \$12,873
\end{onecolentry}

\vspace{0.3 cm}

\begin{onecolentry}
    \textbf{Temple University Awards:}
\end{onecolentry}

\vspace{0.1 cm}

\begin{onecolentry}
    Mar. 2022 Dissertation Completion Grant – \$8,750
\end{onecolentry}

\begin{onecolentry}
    May 2020 First Summer Research Initiative Award – \$6,000
\end{onecolentry}

\begin{onecolentry}
    May 2022 Jay Novik Graduate Student Fellowship – \$5,000 for exceptional performance in the Graduate Mathematics Program
\end{onecolentry}

\begin{onecolentry}
    Summer 2019 NIH Brain Initiative Summer School Funding Award
\end{onecolentry}

\begin{onecolentry}
    Jan. 2019 Mathematics Department Excellence in Teaching Award – \$500
\end{onecolentry}

\begin{onecolentry}
    Nov. 2018 SIAM Recognition – Leadership and coordination of SIAM Chapter activities
\end{onecolentry}

\vspace{0.3 cm}

\begin{onecolentry}
    \textbf{Rowan University Awards:}
\end{onecolentry}

\vspace{0.1 cm}

\begin{onecolentry}
    2015–2016 Certificate of Achievement in Mathematics – For JMM participation and publishing
\end{onecolentry}

\begin{onecolentry}
    2014–2015 Certificate of Achievement in Mathematics – For JMM participation
\end{onecolentry}

\vspace{0.3 cm}

\begin{onecolentry}
    \textbf{Rutgers University Awards:}
\end{onecolentry}

\vspace{0.1 cm}

\begin{onecolentry}
    2003–2007 Edward J. Bloustein Distinguished Scholar
\end{onecolentry}

\section{Presentations}

\begin{onecolentry}
    June 2022 — \textit{“Ultrastructural Neuronal Modeling of Calcium Dynamics Under Transcranial Magnetic Stimulation”}, Doctoral Defense, Temple University, Philadelphia, PA
\end{onecolentry}

\begin{onecolentry}
    Oct. 2021 — \textit{“Neuron Dendritic Spines: Modeling Calcium Communication”}, CST Research Mixer, Temple University, Philadelphia, PA
\end{onecolentry}

\begin{onecolentry}
    May 2021 — \textit{“Hodgkin-Huxley Conductance Based Model: From 1D to 3D”}, Numerical PDEs Course, Temple University, Philadelphia, PA
\end{onecolentry}

\begin{onecolentry}
    Jan. 2021 — \textit{“Applied Mathematics: Modeling Neuronal Electrical and Ion Dynamics”}, Temple University, Philadelphia, PA
\end{onecolentry}

\begin{onecolentry}
    Dec. 2020 — \textit{“PDE Based Image Reconstruction”}, Numerical PDEs Course, Temple University, Philadelphia, PA
\end{onecolentry}

\begin{onecolentry}
    Nov. 2020 — \textit{“Modeling an Action Potential and Neuronal Behavior”}, Graduate Seminar, Temple University, Philadelphia, PA
\end{onecolentry}

\begin{onecolentry}
    Jan. 2020 — \textit{“An Investigation of Spine to Dendrite Calcium Communication”}, Applied Mathematics Seminar, Temple University, Philadelphia, PA
\end{onecolentry}

\begin{onecolentry}
    Oct. 2019 — \textit{“Inner Workings of a Neuron: A Mathematical Perspective”}, Temple University Math Club, Philadelphia, PA
\end{onecolentry}

\begin{onecolentry}
    May 2019 — \textit{“A Walk Through Calculus of Variations”}, Methods in Applied Mathematics Course, Temple University, Philadelphia, PA
\end{onecolentry}

\begin{onecolentry}
    Dec. 2018 — \textit{“Sturm-Liouville Theory: An Example”}, Methods in Applied Mathematics Course, Temple University, Philadelphia, PA
\end{onecolentry}

\begin{onecolentry}
    March 2018 — \textit{“A Table of Definite Integrals from the Marriage of Power and Fourier Series”}, EPaDel Spring Section Meeting, Philadelphia, PA (with T. Osler)
\end{onecolentry}

\begin{onecolentry}
    Fall 2017 — \textit{“Partitions of Equiangular Tight Frames”}, Graduate Student Seminar, Temple University, Philadelphia, PA
\end{onecolentry}

\begin{onecolentry}
    Jan. 2017 — \textit{“Partitions of Equiangular Tight Frames”}, Joint Mathematics Meeting, Atlanta, GA (with H. Nguyen and L. Cao)
\end{onecolentry}

\begin{onecolentry}
    May 2016 — \textit{“Partitions of Equiangular Tight Frames”}, STEM Symposium, Rowan University, Glassboro, NJ (with H. Nguyen and L. Cao)
\end{onecolentry}

\begin{onecolentry}
    April 2016 — \textit{“Frame Partitioning Algorithms”}, Master’s Thesis Presentation, Rowan University, Glassboro, NJ
\end{onecolentry}

\begin{onecolentry}
    May 2015 — \textit{“A Table of Definite Integrals from the Marriage of Power and Fourier Series”}, STEM Symposium, Rowan University, Glassboro, NJ (with T. Osler)
\end{onecolentry}

\begin{onecolentry}
    Aug. 2015 — \textit{“A Table of Definite Integrals from the Marriage of Power and Fourier Series”}, MAA Mathfest, Washington, D.C. (with T. Osler)
\end{onecolentry}

\begin{onecolentry}
    May 2015 — \textit{“A Table of Definite Integrals from the Marriage of Power and Fourier Series”}, Sigma Xi Research Symposium, St. Joseph’s University, Philadelphia, PA (with T. Osler)
\end{onecolentry}

\begin{onecolentry}
    Jan. 2015 — \textit{“A Table of Definite Integrals from the Marriage of Power and Fourier Series”}, Joint Mathematics Meeting, San Antonio, TX (with T. Osler)
\end{onecolentry}

\section{Undergraduate Research}

\begin{twocolentry}{
    Summer 2005
}
    \textbf{Undergraduate Engineering Research}, College of Engineering, Rowan University -- Glassboro, NJ
\end{twocolentry}

\vspace{0.10 cm}
\begin{onecolentry}
    \begin{highlights}
        \item Designed transistor-based replica of leech heart interneuron
        \item Modeled system using Hodgkin-Huxley formalisms for circuit-based neuron simulation
        \item Investigated FPGA experimentation using Mentor Graphics Advantage software
    \end{highlights}
\end{onecolentry}

\vspace{0.2 cm}

\begin{twocolentry}{
    June 2000 – June 2003
}
    \textbf{Undergraduate Research}, College of Humanities and Social Sciences, Rowan University -- Glassboro, NJ
\end{twocolentry}

\vspace{0.10 cm}
\begin{onecolentry}
    \begin{highlights}
        \item Conducted archaeological conservation at the Museum of La Serena, Chile
        \item Designed a statistical program to estimate human stature from archaeological remains
        \item Participated in archaeological site mapping and international research collaboration
    \end{highlights}
\end{onecolentry}

\vspace{0.2 cm}

\begin{twocolentry}{
    June 2004
}
    \textbf{Undergraduate Research (Abroad)}, Universidad Catolica del Norte -- Coquimbo, Chile
\end{twocolentry}

\vspace{0.10 cm}
\begin{onecolentry}
    \begin{highlights}
        \item Translated and analyzed ChemCAD software for shellfish cultivation system design
        \item Studied the aquacultural processes for farming and harvesting shellfish
    \end{highlights}
\end{onecolentry}

\section{Professional Affiliations}

\begin{onecolentry}
    Sep. 2018 – June 2022 — Temple University SIAM Chapter President
\end{onecolentry}

\begin{onecolentry}
    Sep. 2018 – June 2022 — Temple University SIAM Chapter Member
\end{onecolentry}

\begin{onecolentry}
    Sep. 2018 – June 2022 — Society for Industrial and Applied Mathematics (SIAM) Member and Journal Subscriber
\end{onecolentry}

\begin{onecolentry}
    Sep. 2017 – June 2022 — American Mathematical Society (AMS) Member
\end{onecolentry}

\begin{onecolentry}
    Sep. 2017 – June 2022 — Mathematical Association of America (MAA) Member
\end{onecolentry}

\begin{onecolentry}
    Sep. 2015 – June 2017 — Rowan University Pi Mu Epsilon Chapter Member
\end{onecolentry}

\begin{onecolentry}
    Sep. 2015 – June 2017 — Pi Mu Epsilon National Mathematics Honor Society Member
\end{onecolentry}

\begin{onecolentry}
    Sep. 2008 – June 2017 — New Jersey Education Association (NJEA) Member
\end{onecolentry}

\begin{onecolentry}
    Sep. 2008 – June 2017 — Clearview Regional Education Association Union Member
\end{onecolentry}

\begin{onecolentry}
    Sep. 2006 – Sep. 2008 — Institute of Electrical and Electronics Engineers (IEEE) Member and Journal Subscriber
\end{onecolentry}

\section{Hobbies}

\begin{onecolentry}
    \textbf{Reading:} Neuroscience, mathematics, science fiction, history, poetry, biographies, philosophy, high fantasy
\end{onecolentry}

\vspace{0.2 cm}

\begin{onecolentry}
    \textbf{Creative pursuits:} Writing mathematical articles and poetry, drawing, painting
\end{onecolentry}

\vspace{0.2 cm}

\begin{onecolentry}
    \textbf{Fitness:} Running, weight lifting, hiking
\end{onecolentry}

\vspace{0.2 cm}

\begin{onecolentry}
    \textbf{Sports and entertainment:} Attending MLB and NFL games (Orioles, Phillies, Ravens, Eagles), video games, board games
\end{onecolentry}

\vspace{0.2 cm}

\begin{onecolentry}
    \textbf{Leisure and travel:} Exploring restaurants, breweries, wineries; traveling within the U.S. and abroad
\end{onecolentry}


    

\end{document}